\documentclass{amsart}


%% Don't edit this part!
% For revision control
\usepackage{rcs-multi}
\rcsid{$Id$}
\rcsid{$Header$}
\rcskwsave{$Author$}
\rcskwsave{$Date$} 
\rcskwsave{$Revision$}
%%\rcsRegisterAuthor{devangel}{Dennis Jos{\'e} Evangelista}
\rcsRegisterAuthor{devangel}{Dennis J. Evangelista}
\rcsRegisterAuthor{wxy}{WANG Xiaoyu}

\usepackage{graphicx}
\usepackage[usenames,dvipsnames]{color}
%\usepackage{makeidx} % incompatible with ams art
\usepackage{siunitx}
\DeclareMathOperator*{\argmin}{\arg\!\min}
\usepackage{multirow}
\usepackage{colortbl}

% PDF metadata
\usepackage{hyperref}
\hypersetup{pdftitle={Applied math warmup: solving the heat equation}}
\hypersetup{pdfauthor={Xiaoyu Wang and Dennis Evangelista}}
\hypersetup{pdfsubject={biology}}
\hypersetup{pdfkeywords={biomechanics, applied math, heat equation, diffusion, finite difference method}}
\hypersetup{colorlinks=true,citecolor=Violet,linkcolor=Blue,urlcolor=Red}








\title{Applied math warmup: solving the heat equation}
\author{Xiaoyu Wang and Dennis Evangelista}
\address{Department of Integrative Biology, UC Berkeley}
\email{devangel@berkeley.edu}
\thanks{Thanks to Christy Hamlet and Lindsay Waldrop for discussions on this topic. Also the methods used here were first introduced to me by Dave Hay and Jim Turso.}
\date{\today}

\begin{document}
\begin{abstract}
This is an example of how to use \LaTeX to write up some notes about math. 
\end{abstract}
\maketitle
\tableofcontents

\section{Introduction}
This is a citation \cite{Baker:2012}

This is an equation:
\begin{equation}
\dot{q} = - k \nabla T
\label{eq:NLC}
\end{equation}.

This is another equation:
\begin{equation}
\frac{\partial T}{\partial t} = 
k \nabla^2 T + \dot{q}_{gen}
\label{eq:heat}
\end{equation}

I can reference equations~\ref{eq:NLC} and \ref{eq:heat} this way. 

\section{Methods}
\subsection{Submethod one}
\subsection{Submethod two}
\subsection{Submethod three}

\section{Results}
\section{Discussion}

% AMS style references
\bibliographystyle{amsplain}
\bibliography{references/warmup}
\end{document}
